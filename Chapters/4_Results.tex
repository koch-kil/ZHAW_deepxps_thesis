% Indicate the main file. Must go at the beginning of the file.
% !TEX root = ../main.tex

%----------------------------------------------------------------------------------------
% CHAPTER TEMPLATE
%----------------------------------------------------------------------------------------


\chapter{Results and discussion} % Main chapter title

\label{Chapter4}



%----------------------------------------------------------------------------------------
% SECTION 1
%----------------------------------------------------------------------------------------
\section{Qualitative elemental identification of two-layered systems}
\subsubsection{Elemental identification}
% model performance
The model performance for the qualitative elemental identification is shown in Table \ref{tab:acc_qual}. The categorical accuracies were computed for the individual cont. and clean datasets, and the mixcont dataset.

\begin{table}[H]
    \centering
    \begin{tabular}{c|c|c|c|c|c}
        Dataset & Layer & Model   & No. Parameters & Validation Dataset*  & Test Dataset*    \\
        \hline
        mixcont & top   & CNN     &                & 85.32 \%              &        \% \\
                &       & CNN-DCT &                &       \%              &         \% \\
                &       & CBAM    &                &       \%              &         \% \\
                &       & ViT     &                &  91.66\%              &   99.37\% \\
                & bot   & CNN     &                &  73.85\%              &          \%  \\
                &       & CNN-DCT &                &       \%              &          \%  \\
                &       & CBAM    &                &       \%              &         \% \\
                &       & ViT     &                &       \%              &           \% \\
    \end{tabular}
    \caption{Categorical Accuracies, AUC and Number of Parameters of the models}
    \label{tab:acc_qual}
\end{table}


% experimental data (AG_AG etc.)
From the experimental data, $\nelementalspectra$ elemental spectra were used for the qualitative elemental prediction. 


\section{Quantitative elemental identification of one-layered systems}

The quantitative models' performance was trained on the mixcont + multi dataset.
% confusion matrix on test-data

% other visualisation and accuracy figures for classification
\begin{table}[H]
    \centering
    \begin{tabular}{c|c|c|c|c|c}
        Dataset & Model   & No. Parameters & Categorical Accuracy      \\
        \hline
        mixcont & CNN     &                 &             \% \\
                & CNN-DCT &                 &             \% \\
                & CBAM    &                 &             \% \\
                & ViT     &                 &             \% \\
                & CNN     &                 &             \%  \\
                & CNN-DCT &                 &             \%  \\
                & CBAM    &                 &             \% \\
                & ViT     &                 &             \% \\
    \end{tabular}
    \caption{Categorical Accuracies, AUC and Number of Parameters of the models}
    \label{tab:acc_qual}
\end{table}

% plot with quantification (imshow eg.)\textbf{}

% plot visual attention feature

% experimental data if possible

\section{Depth profiling of two-layered systems}
