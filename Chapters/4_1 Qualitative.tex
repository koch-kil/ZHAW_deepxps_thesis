%----------------------------------------------------------------------------------------
% SECTION 1
%----------------------------------------------------------------------------------------
\section{Qualitative elemental identification of bilayer systems}
\subsubsection{Elemental identification}
% model performance
The model performance for the qualitative elemental identification is shown in Table \ref{tab:acc_qual}. The categorical accuracies were computed for the individual cont. and clean datasets, and the mixcont dataset. It should be noted that - due to the small amount of information from the bottom layer - we expect a lower performance compared to the top layer.

\begin{table}[H]
    \centering
    \begin{tabular}{c|c|c|c|c|c}
        Dataset & Layer & Model   & No. Parameters & Validation Dataset*  & Test Dataset*    \\
        \hline
        mixcont & top   & CNN     &                & 85.32 \%              &        \% \\
                &       & CNN-DCT &                &       \%              &         \% \\
                &       & CBAM    &                &       \%              &         \% \\
                &       & ViT     &                &  91.66\%              &   99.37\% \\
                & bot   & CNN     &                &  73.85\%              &          \%  \\
                &       & CNN-DCT &                &       \%              &          \%  \\
                &       & CBAM    &                &       \%              &         \% \\
                &       & ViT     &                &       \%              &           \% \\
    \end{tabular}
    \caption{Categorical Accuracies, and number of Parameters of the models in respect to dataset and layer}
    \label{tab:acc_qual}
\end{table}


% experimental data (AG_AG etc.)
From the experimental data, $\nelementalspectra$ elemental spectra were used for the qualitative elemental prediction. 

\subsubsection{Depth and gradient determination of native oxides and elements}
% As there's almost no test data this is experimental
% model performance


\begin{table}[H]
    \centering
    \begin{tabular}{c|c|c|c|c|c}
        Dataset & Model   & No. Parameters & Validation Dataset*  & Test Dataset*    \\
        \hline
        mixcont+oxides& CNN     &                &        \%             &             \% \\
               & CNN-DCT &                &       \%              &             \% \\
               & CBAM    &                &       \%              &             \% \\
               & ViT     &                &       \%              &             \% \\
               & CNN     &                &       \%              &             \%  \\
               & CNN-DCT &                &       \%              &             \%  \\
               & CBAM    &                &       \%              &             \% \\
               & ViT     &                &       \%              &             \% \\
    \end{tabular}
    \caption{ and number of Parameters of the models in respect to dataset and layer}
    \label{tab:acc_depth}
\end{table}
% experimental data (AG_AG etc.)
