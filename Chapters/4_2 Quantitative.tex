%----------------------------------------------------------------------------------------
% SECTION 2
%----------------------------------------------------------------------------------------
\section{Elemental quantification of single layer systems}

% model performance
The model performance for the quantitative assessment of chemical composition is shown in Table \ref{tab:acc_quant}. The mean squared errors were computed for the multi dataset for each model.
% experimental data (AG_AG etc.)
From all experimental data, $\nmultispectra$ elemental spectra were used for the quantitative prediction. 

\begin{table}[H]
    \centering
    \begin{tabular}{c|c|c|c|c|c}
       Dataset & Model   & No. Parameters & Validation Dataset*  & Test Dataset*    \\
        \hline
        multi  & CNN     &                &        \%              &         \% \\
               & CNN-DCT &                &       \%              &          \% \\
               & CBAM    &                &       \%              &          \% \\
               & ViT     &                &       \%              &          \% \\
    \end{tabular}
    \caption{Number of Parameters and Threshold accuracy of the models}
    \label{tab:acc_quant}
\end{table}


In contrast to previous studies on deep-learning assisted quantitative XPS analysis \cite{drera_deep_2019}, no relative sensitivity factors were applied in this approach. Furthermore, a significantly smaller training dataset of 30k versus the 100k which were used in their work. Lastly, the test data spectra they used were collected in experimental conditions and the training dataset was simulated according to the known conditions.

% plot visual attention feature