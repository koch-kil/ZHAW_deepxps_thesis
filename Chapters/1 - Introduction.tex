% Indicate the main file. Must go at the beginning of the file.
% !TEX root = ../main.tex

%----------------------------------------------------------------------------------------
% CHAPTER 1
%----------------------------------------------------------------------------------------



\chapter{Introduction} % Main chapter title
\label{Chapter1} % For referencing the chapter elsewhere, use \ref{Chapter1} 

%----------------------------------------------------------------------------------------

% Define some commands to keep the formatting separated from the content
% Placing such commands in the preamble is a good idea.
\newcommand{\keyword}[1]{\textbf{#1}}
\newcommand{\tabhead}[1]{\textbf{#1}}
\newcommand{\code}[1]{\texttt{#1}}
\newcommand{\file}[1]{\texttt{\bfseries#1}}
\newcommand{\option}[1]{\texttt{\itshape#1}}


%----------------------------------------------------------------------------------------

% First part: background, situation, problem, the current state of the art or science
% primary literature

X-ray photoelectron spectroscopy (XPS) is a spectroscopic technique that allows researchers to study the chemical composition of surfaces. It is based on the fundamental principle of the photoelectric effect first described by Einstein \cite{einstein_uber_1905} in 1905. Later, Rutherford 

With the information depth of XPS at $\sim$10 nm, it is possible to probe surfaces and - depending on the sample - ultra thin films.
Thus, XPS is used across multiple fields in the material science domain to determine surface properties, such as corrosion, welding, fatigue, oxidation, coatings, adhesives, catalysis and semiconductors \cite{noauthor_x-ray_nodate}.

%----------------------------------------------------------------------------------------

% Second part: research gap or the relevance of the work for research
% The aim is to show which areas have been neglected so far or have recently gained importance.
Despite the widespread use of XPS analysis due to improved user friendliness of the technique, users often misinterpret data - especially in quantitative determination \cite{shard_practical_2020}

% why is gradient analysis interesting - it's not been done so far but very important - usually determined with other techniques?


%----------------------------------------------------------------------------------------

% Third part: objective of the work, formulate your own task as precisely as possible
% If necessary in the form of form of testable hypotheses




%----------------------------------------------------------------------------------------

%For more detailed presentations of the current state of the art or theory development or a literature review, a separate chapter is useful