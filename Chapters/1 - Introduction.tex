% Indicate the main file. Must go at the beginning of the file.
% !TEX root = ../main.tex

%----------------------------------------------------------------------------------------
% CHAPTER 1
%----------------------------------------------------------------------------------------



\chapter{Introduction} % Main chapter title
\label{Chapter1} % For referencing the chapter elsewhere, use \ref{Chapter1} 

%----------------------------------------------------------------------------------------

% Define some commands to keep the formatting separated from the content
% Placing such commands in the preamble is a good idea.
\newcommand{\keyword}[1]{\textbf{#1}}
\newcommand{\tabhead}[1]{\textbf{#1}}
\newcommand{\code}[1]{\texttt{#1}}
\newcommand{\file}[1]{\texttt{\bfseries#1}}
\newcommand{\option}[1]{\texttt{\itshape#1}}


%----------------------------------------------------------------------------------------

% First part: background, situation, problem, the current state of the art or science
% primary literature


The chemical composition of a solid is one of the parameters defining its physical properties. However, in many cases, the surface of a solid differs from the bulk (volume) properties. This may be unintentionally, e.g., by oxidation and/or further corrosion, or intentionally by depositing a surface with a functional coating. 
Various surface analysis techniques can be used to investigate the surface properties such as the composition or electronic state.


X-ray photoelectron spectroscopy (XPS) is a spectroscopic technique is one of the surface analysis techniques. It is based on the fundamental principle of the photoelectric effect first described by Einstein \cite{einstein_uber_1905} in 1905. 

% TODO Later, Rutherford 

With the information depth of XPS at $\sim$10 nm, it is possible to probe surfaces and - depending on the sample - ultra thin films.
Thus, XPS is used across multiple fields in the material science domain to determine surface properties, such as corrosion, welding, fatigue, oxidation, coatings, adhesives, catalysis and semiconductors \cite{noauthor_x-ray_nodate}.



%----------------------------------------------------------------------------------------

% Second part: research gap or the relevance of the work for research
% The aim is to show which areas have been neglected so far or have recently gained importance.
% what makes xps so complicated? cross sections (probabilities) for photoemission
Despite the widespread use of XPS analysis due to recent improved user friendliness of the technique, users often misinterpret data - especially in quantitative determination \cite{shard_practical_2020}. 

% TODO However, the as the information  


%ja, nee, it is not the misintepretaation, it is more, that (i) XPS comes to its limits for complex (multi-elemental) samples, and (ii) XP-spectra contain more information than usually extracted.... Dann musst Du natürlich ein zwei Beispiele dafür geben...


With this powerful technique, we could potentially probe into the surface of samples and obtain the concentration of elements as a function of depth - also called depth profile. Many adaptations of the XPS technique, as well as modelling approaches have been developed to reach this goal \cite{zemek_non-destructive_2019, zborowski_improved_2022, zborowski_comparison_2022, noauthor_energy_2010}. However, most techniques available are cost-intensive and require additional information to the spectrum which must be obtained by other analyses or models.  


% why is gradient analysis interesting - it's not been done so far but very important - usually determined with other techniques?

%----------------------------------------------------------------------------------------

% Third part: objective of the work, formulate your own task as precisely as possible
% If necessary in the form of testable hypotheses
Because there exist limited data from publicly accessible databases for XPS spectra, the simulation software Sessa was used to generate >150k spectra for the training set. Although we rarely find perfectly homogeneous layers when analyzing samples with XPS, this assumption was made due to the high complexity implied when trying to model inhomogeneous layers.

% history of deep learning for spectroscopy - and deep learning patterns in general

In this study, we demonstrate the utilization of deep learning for the purpose of identifying and quantifying XPS survey spectra. Further, we show how various deep learning models perform on the prediction of gradients and depth composition of samples based on XPS survey spectra.

We formualte three tasks to be investigated in this work using a variety of deep learning models.
\begin{enumerate}
    \item Predict composition of elemental bilayer systems
    \item Predict the depth of the surface layer in elemental bilayer systems 
    \item Predict quantitative composition of single layer systems
\end{enumerate}
And further there is an underlying hypothesis rising from the approach used in this work that is, whether we can solve these tasks with simulation data based neural network approaches.

% detection limit --> 


%----------------------------------------------------------------------------------------

%For more detailed presentations of the current state of the art or theory development or a literature review, a separate chapter is useful