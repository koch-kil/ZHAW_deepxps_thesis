% !TEX root = ../main.tex

%----------------------------------------------------------------------------------------
% ABSTRACT PAGE
%----------------------------------------------------------------------------------------
\begin{abstract}
\addchaptertocentry{\abstractname} % Add the abstract to the table of contents
Surface analysis for the determination of chemical composition of a solid palys an important role in material sciences. X-ray photoelectron spectroscopy is widely applied for such structural analysis, giving information beyond the outermost surface, as it probes deeper to approximately 10 nanometers depth. Analysis of the spectrum to model the solid under investigation is complex, as many effects complain the inference, such as inelastic and elastic scattering and sample contamination. This work applies deep learning models to three main problems; qualitative identification, quantitative identification and depth profiling.
To overcome the lack of available labelled XPS data, a simulation software was used to obtain >300k XPS spectra as training data. Four different models architectures were applied to the main problems; convolutional neural network (CNN), convolutional dct-informed neural network (CNN-DCT), Residual convolutional block attention model (CBAM) and Vision Transformer (ViT). 
\end{abstract}


%----------------------------------------------------------------------------------------
% German ABSTRACT PAGE
%----------------------------------------------------------------------------------------
\begin{extraAbstract}
\addchaptertocentry{\extraabstractname} % Add the abstract to the table of contents

Die Zusammenfassung entspricht einer Miniaturversion des gesamten Dokuments. Gliedere sie ähnlich: Beginne mit dem Kontext und der Motivation für das Projekt, einer kurzen Beschreibung der Methode und der verfügbaren Daten, Ihren Ergebnissen und den Schlussfolgerungen. Beschränke dich auf eine Seite!    
\end{extraAbstract}
